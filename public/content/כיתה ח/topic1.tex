# כיתה ח' – משוואות עם נעלם אחד

### הצצה לעולם המשוואות

דמיינו שאתם מגלים חידות שבהן המספרים מוסתרים, ואתם צריכים לפענח מה הם בעזרת רמזים. משוואה היא בדיוק כזו חידה! עכשיו נלמד איך לפתור משוואות עם נעלם אחד בדרך פשוטה וכיפית.

---

### למה זה חשוב? (מוטיבציה חמודה)

נניח שקניתם 3 חולצות זהות וסכום הקנייה הכולל היה 90 שקלים. כמה עולה כל חולצה? בעזרת משוואות תוכלו לפתור את הבעיה בקלות. במקום לנחש, פשוט כותבים: 3x = 90, ואז פותרים כדי לגלות את המחיר של כל חולצה.

---

## נושא 1: מהי משוואה?

### מה זה בעצם?

משוואה היא משפט מתמטי שמראה ששני צדדים שווים. במשוואה יש "נעלם", כלומר מספר שאנחנו צריכים למצוא.

### דוגמה פשוטה:

אם יש לנו $x + 5 = 12$, המשמעות היא שהמספר שצריך להוסיף ל-5 כדי לקבל 12 הוא $x$. 

### איך פותרים משוואה?

כדי לפתור משוואה, אנחנו מבצעים את הפעולות ההפוכות כדי "לבודד" את הנעלם.

---

### דוגמה פתורה:

**פתרו את המשוואה $x + 7 = 15$.**

1. כדי לבודד את $x$, מחסרים 7 משני הצדדים:
   $x + 7 - 7 = 15 - 7$
2. מקבלים: $x = 8$

**התשובה היא:** $x = 8$.

---

### תרגיל פתור:

**פתרו את המשוואה $2x = 14$.**

1. כדי לבודד את $x$, מחלקים את שני הצדדים ב-2:
   $2x \div 2 = 14 \div 2$
2. מקבלים: $x = 7$

**התשובה היא:** $x = 7$.

---

### תרגילים:

1. פתרו את $x - 3 = 10$.
   - מוסיפים 3 לשני הצדדים: $x - 3 + 3 = 10 + 3$
   - מקבלים: $x = 13$

2. פתרו את $4x = 32$.
   - מחלקים את שני הצדדים ב-4: $4x \div 4 = 32 \div 4$
   - מקבלים: $x = 8$

3. פתרו את $x + 9 = 20$.
   - מחסרים 9 משני הצדדים: $x + 9 - 9 = 20 - 9$
   - מקבלים: $x = 11$

---

## לסיכום:

משוואות הן כלי נהדר שמאפשר לנו לפתור בעיות יומיומיות בצורה מסודרת ומהירה. כל מה שצריך זה להבין את הפעולה ההפוכה ולבודד את הנעלם.

### מוכנים להמשיך לנושא הבא? יחד נמשיך ללמוד ולהשתפר!

