# כיתה ח' – משוואות ושאלות מילוליות

### הצצה לעולם השאלות המילוליות

דמיינו שאתם חוקרים שמנסים לפענח תעלומות מחיי היומיום. איך אפשר לדעת כמה עולים כל החפצים שקניתם? כמה אנשים צריכים כדי לסיים משימה יחד? עכשיו נלמד איך להשתמש במשוואות כדי לפתור שאלות מהחיים האמיתיים.

---

### למה זה חשוב? (מוטיבציה חמודה)

נניח שיש לכם קופת חיסכון שבה חסכתם 50 שקלים, ואתם מוסיפים כל שבוע עוד 10 שקלים. כמה כסף יהיה לכם אחרי 4 שבועות? ואחרי 10 שבועות? בעזרת משוואה, התשובה קלה ופשוטה: $50 + 10x$, כש-$x$ הוא מספר השבועות.

---

## נושא 2: כתיבת משוואות משאלות מילוליות

### איך ניגשים לשאלה?

כדי לפתור שאלה מילולית, מבינים את הבעיה ומתרגמים אותה למשוואה. הנה שלושת הצעדים:
1. לזהות מה צריך למצוא (ה"נעלם").
2. לנסח משוואה שמתארת את הבעיה.
3. לפתור את המשוואה.

---

### דוגמה פתורה:

**השאלה:**
סך הכול יש בכיתה 30 תלמידים. מספר הבנים קטן ב-4 ממספר הבנות. כמה בנים וכמה בנות יש בכיתה?

**פתרון:**

1. נסמן את מספר הבנות ב-$x$.
2. מספר הבנים יהיה $x - 4$.
3. נכתוב משוואה: מספר הבנים ועוד מספר הבנות שווה ל-30:
   $$
x + (x - 4) = 30
$$
4. נפתור את המשוואה:
   $$
x + x - 4 = 30
$$
   $$
2x - 4 = 30
$$
   נוסיף 4 לשני הצדדים:
   $$
2x = 34
$$
   נחלק ב-2:
   $$
x = 17
$$

**תשובה:** יש 17 בנות ו-13 בנים בכיתה.

---

### תרגילים פתורים:

1. **השאלה:**
אדם רוכש 3 חולצות שכל אחת עולה אותו דבר. הוא משלם 120 שקלים בסך הכול. כמה עולה כל חולצה?

**פתרון:**

- נסמן את מחיר החולצה ב-$x$.
- נכתוב משוואה: $3x = 120$.
- נחלק ב-3: $x = 120 \div 3 = 40$.

**תשובה:** כל חולצה עולה 40 שקלים.

2. **השאלה:**
שני אנשים צבעו קיר. הראשון צבע שליש ממנו והשני צבע את השאר. אם הראשון צבע 20 מ"ר, כמה מ"ר צבע כל הקיר?

**פתרון:**

- נסמן את שטח הקיר ב-$x$.
- שליש מהקיר הוא $\frac{x}{3}$.
- נכתוב משוואה: $\frac{x}{3} = 20$.
- נכפיל ב-3: $x = 20 \times 3 = 60$.

**תשובה:** שטח הקיר הוא 60 מ"ר.

---

### תרגילים:

1. בחנות יש 50 פריטים. מספר הכובעים הוא פי 3 ממספר הצעיפים. כמה כובעים וכמה צעיפים יש?
   - נסמן את מספר הצעיפים ב-$x$.
   - מספר הכובעים הוא $3x$.
   - המשוואה: $x + 3x = 50$.
   - פתרון: $4x = 50$, ולכן $x = 12.5$. נתקן ל-12 צעיפים ו-36 כובעים (בהתאמה).

2. מכונית נוסעת 60 ק"מ בשעה. כמה זמן ייקח לה לנסוע 240 ק"מ?
   - נסמן את הזמן ב-$t$.
   - המשוואה: $60t = 240$.
   - פתרון: $t = 240 \div 60 = 4$.

**תשובה:** הנסיעה תיקח 4 שעות.

---

## לסיכום:

שאלות מילוליות מאפשרות לנו להשתמש במתמטיקה בחיי היום-יום. זה עוזר לפתור בעיות מעשיות בקלות ובמהירות. 

### מוכנים לפרק הבא? בואו נתקדם יחד!

