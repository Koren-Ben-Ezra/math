# מבוא לכל הנושאים

### ברוכים הבאים לעולם המתמטיקה של כיתה ז'

מתמטיקה היא כמו ארגז כלים מלא בכלים שימושיים. כל נושא שתלמדו השנה הוא כלי חדש שיעזור לכם לפתור בעיות ולגלות דברים מרתקים על העולם. בפרק זה נעשה היכרות עם כל הנושאים המרכזיים של השנה, כולל דוגמאות קטנות שיעזרו לכם להבין כל נושא.

---

## נושא 1: משתנים וביטויים אלגבריים

משתנים הם כמו קופסאות ריקות שאפשר למלא במספרים, וביטויים אלגבריים הם כמו משפטים שמתארים חישובים עם הקופסאות האלה.

### דוגמה:
אם $x$ הוא מספר התפוחים שיש לכם, הביטוי $x + 3$ אומר "מספר התפוחים שיש לכם ועוד 3".

- אם $x = 5$, אז $x + 3 = 8$.

---

## נושא 2: פעולות חשבון וחזקות

פעולות החשבון (חיבור, חיסור, כפל וחילוק) עוזרות לנו לפתור בעיות בקלות. חזקות הן דרך לקצר חישוב של כפל חוזר.

### דוגמה:
$2^3$ פירושו $2 \times 2 \times 2$, והתוצאה היא $8$.

---

## נושא 3: סדר פעולות החשבון

סדר פעולות החשבון אומר לנו באיזה סדר לפתור תרגילים שיש בהם כמה פעולות. הסדר הוא: סוגריים, חזקות, כפל וחילוק, חיבור וחיסור.

### דוגמה:
בתרגיל $2 + 3 \times (4 - 1)$:
1. פותרים את הסוגריים: $4 - 1 = 3$
2. מכפילים: $3 \times 3 = 9$
3. מחברים: $2 + 9 = 11$

התוצאה היא $11$.

---

## נושא 4: משוואות בנעלם אחד

משוואה היא חידה מתמטית שבה אנחנו מנסים לגלות את הערך של הנעלם.

### דוגמה:
אם $x + 3 = 7$, נחסיר 3 משני הצדדים ונקבל $x = 4$.

---

## נושא 5: מספרים מכוונים

מספרים מכוונים הם מספרים שיכולים להיות חיוביים או שליליים. הם עוזרים לנו למדוד דברים כמו טמפרטורות מעל ומתחת לאפס.

### דוגמה:
$5 - (-3)$ = $5 + 3 = 8$.

---

## נושא 6: מבוא לגיאומטריה בסיסית

גיאומטריה עוסקת בצורות, שטחים, היקפים ועוד. נלמד לחשב היקף ושטח של צורות בסיסיות.

### דוגמה:
אם אורך מלבן הוא $5$ ורוחבו הוא $3$, אז היקפו הוא $2 \times (5 + 3) = 16$.

---

## נושא 7: נפחים של תיבות

גיאומטריה עוסקת גם בנפח. בפרק זה נלמד לחשב נפחים של תיבות בצורות בסיסיות.

### דוגמה:

אם אורך התיבה הוא 5, רוחבה 3 וגובהה 2, אז הנפח הוא $5 \times 3 \times 2 = 30$.

---

## נושא 8: פונקציות בסיסיות

פונקציות מתארות קשרים בין שני דברים. נלמד לייצג פונקציות בעזרת טבלאות, גרפים וביטויים.

### דוגמה:
אם הפונקציה אומרת שכמות הכסף שתקבלו היא $y = 2x$ וש-$x$ הוא מספר השעות שעבדתם, אז אם עבדתם 3 שעות, תרוויחו $y = 2 \times 3 = 6$ שקלים.

---

### לסיכום:

במהלך השנה תכירו כלים חשובים במתמטיקה שיעזרו לכם להבין את העולם טוב יותר. כל נושא קשור לאחרים, וביחד הם יוצרים תמונה שלמה ומרתקת. מוכנים להתחיל? בואו נצלול לנושא הראשון!

