# מספרים חיוביים ושליליים

## הקדמה לנושא

מספרים חיוביים ושליליים, הידועים גם כמספרים מכוונים, עוזרים לנו להבין ולמדוד תופעות שונות כמו טמפרטורות, חובות וכדומה. המספרים הללו מאפשרים לנו לעבוד עם כיוונים שונים, כמו למעלה ולמטה או ימין ושמאל.

### למה זה חשוב?
1. **טמפרטורות**: להבין מה קורה כשקר מתחת לאפס.
2. **כספים**: לייצג חובות או רווחים.
3. **תנועה**: למדוד מיקום ביחס לנקודת התחלה, כמו על ציר.

---

## מושגי יסוד

### מהו מספר מכוון?
מספר מכוון הוא מספר שמצביע על כיוון מסוים:
- **מספרים חיוביים**: מספרים שמייצגים גובה, חום, רווח, וכדומה (לדוגמה: 5, 12).
- **מספרים שליליים**: מספרים שמייצגים ירידה, קור, חוב, וכדומה (לדוגמה: $-3$, $-8$).

#### דוגמה:
- טמפרטורה של $5^{\circ}C$ מייצגת חום מעל האפס.
- טמפרטורה של $-2^{\circ}C$ מייצגת קור מתחת לאפס.

### מהו ציר המספרים?
ציר המספרים הוא קו ישר שמסומן במספרים:
- אפס נמצא באמצע.
- המספרים החיוביים נמתחים ימינה (1, 2, 3...).
- המספרים השליליים נמתחים שמאלה ($-1$, $-2$, $-3$...).

#### דוגמה:
```
... -3   -2   -1   0   1   2   3 ...
```

---

## פעולות חשבון עם מספרים מכוונים

### חיבור מספרים מכוונים

#### כללים:
1. כאשר שני מספרים הם באותו סימן (שניהם חיוביים או שניהם שליליים):
   - מוסיפים את הערכים ושומרים על הסימן.
   - דוגמה: $3 + 2 = 5$, $-4 + (-3) = -7$.

2. כאשר המספרים בעלי סימנים שונים:
   - מחסרים את הערכים הגדולים ומוסיפים את הסימן של הערך הגדול יותר.
   - דוגמה: $5 + (-3) = 2$, $-7 + 4 = -3$.

#### דוגמה מפורטת:
משימה: $-6 + 8$
1. הערכים הם $6$ ו-$8$.
2. נחשב $8 - 6 = 2$.
3. מכיוון שהמספר הגדול הוא חיובי, התוצאה היא $2$.


### חיסור מספרים מכוונים

#### כללים:
1. הפוך את הסימן של המספר השני (המספר שמחסרים אותו).
2. בצע חיבור לפי הכללים של חיבור מספרים מכוונים.

#### דוגמה מפורטת:
משימה: $3 - (-5)$
1. שנה את סימן $-5$ ל-$+5$.
2. הפוך את התרגיל ל-$3 + 5$.
3. תוצאה: $8$.

#### דוגמה נוספת:
משימה: $-4 - 6$
1. שנה את הסימן של $6$ ל-$-6$.
2. הפוך את התרגיל ל-$-4 + (-6)$.
3. תוצאה: $-10$.

---

### כפל מספרים מכוונים

#### כללים:
1. כאשר שני המספרים הם באותו סימן (שניהם חיוביים או שניהם שליליים):
   - התוצאה היא חיובית.
   - דוגמה: $3 \times 2 = 6$, $(-4) \times (-5) = 20$.

2. כאשר המספרים בעלי סימנים שונים:
   - התוצאה היא שלילית.
   - דוגמה: $-3 \times 2 = -6$, $4 \times (-7) = -28$.

#### דוגמה מפורטת:
משימה: $-6 \times (-3)$
1. שני המספרים שליליים.
2. התוצאה היא $18$ כי $6 \times 3 = 18$ והתוצאה חיובית.

#### דוגמה נוספת:
משימה: $-7 \times 4$
1. המספרים בעלי סימנים שונים.
2. התוצאה היא $-28$.

---

### חילוק מספרים מכוונים

#### כללים:
1. כאשר שני המספרים הם באותו סימן (שניהם חיוביים או שניהם שליליים):
   - התוצאה היא חיובית.
   - דוגמה: $6 \div 2 = 3$, $(-12) \div (-3) = 4$.

2. כאשר המספרים בעלי סימנים שונים:
   - התוצאה היא שלילית.
   - דוגמה: $-10 \div 2 = -5$, $8 \div (-4) = -2$.

#### דוגמה מפורטת:
משימה: $-20 \div (-5)$
1. שני המספרים שליליים.
2. התוצאה היא $4$ כי $20 \div 5 = 4$ והתוצאה חיובית.

#### דוגמה נוספת:
משימה: $15 \div (-3)$
1. המספרים בעלי סימנים שונים.
2. התוצאה היא $-5$.

---

## תרגילים לדוגמה

### חיבור:
1. $7 + (-3)$
2. $-5 + (-6)$
3. $-8 + 12$

### חיסור:
4. $10 - (-4)$
5. $-7 - 3$
6. $-2 - (-5)$

### כפל:
7. $-3 \times 6$
8. $-4 \times (-7)$
9. $5 \times (-2)$

### חילוק:
10. $-18 \div 6$
11. $-24 \div (-8)$
12. $10 \div (-5)$

---

## פתרונות לתרגילים

1. $7 + (-3) = 4$
2. $-5 + (-6) = -11$
3. $-8 + 12 = 4$
4. $10 - (-4) = 10 + 4 = 14$
5. $-7 - 3 = -7 + (-3) = -10$
6. $-2 - (-5) = -2 + 5 = 3$
7. $-3 \times 6 = -18$
8. $-4 \times (-7) = 28$
9. $5 \times (-2) = -10$
10. $-18 \div 6 = -3$
11. $-24 \div (-8) = 3$
12. $10 \div (-5) = -2$

---

## לסיכום

הבנת מספרים מכוונים היא מיומנות חשובה לחיים. בין אם מדובר בטמפרטורות, חובות או תנועות, המספרים החיוביים והשליליים עוזרים לנו לייצג את העולם בצורה מדויקת וברורה. ככל שתתרגלו יותר, תבינו את הקשר בין המספרים ותפתרו בעיות בקלות!
