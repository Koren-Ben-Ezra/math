# נפחים של תיבות: מבוא לגיאומטריה תלת-ממדית

## הקדמה לנושא

גיאומטריה תלת-ממדית עוסקת בצורות שיש להן נפח, כמו תיבות, כדורים, פירמידות ועוד. נתחיל עם אחד הצורות הבסיסיות ביותר: התיבה. התיבה היא גוף תלת-ממדי שיש לה שישה צדדים מלבניים.

### למה זה חשוב?
1. **חיי היומיום**: להבין כמה מקום יש בקופסה או כמה מים ייכנסו למיכל.
2. **בנייה והנדסה**: חישוב נפח של חדרים או מבנים.
3. **לוגיסטיקה**: תכנון קיבולת של ארגזים או מכולות.

---

## מהי תיבה?

### הגדרה
תיבה היא גוף תלת-ממדי בעל שישה פאות, שכל אחת מהן היא מלבן. לתיבה יש:
1. **אורך**: המרחק בין שתי פאות מקבילות.
2. **רוחב**: המרחק בין הפאות האחרות.
3. **גובה**: המרחק בין הפאה התחתונה לפאה העליונה.

#### דוגמה:
דמיינו קופסת נעליים. האורך הוא הצד הארוך של הקופסה, הרוחב הוא הצד הקצר, והגובה הוא המרחק מהתחתית למכסה.

---

## חישוב נפח של תיבה

### נוסחה לחישוב נפח
הנפח של תיבה מחושב על ידי הכפלת האורך, הרוחב והגובה:
\[
נפח = אורך \times רוחב \times גובה
\]

#### דוגמה מפורטת:
נניח שתיבה בעלת אורך 10 ס\"מ, רוחב 5 ס\"מ וגובה 8 ס\"מ:
\[
נפח = 10 \times 5 \times 8 = 400 \text{ ס\"מ מעוקב}
\]

#### דוגמה נוספת:
תיבה בעלת אורך 6 מ', רוחב 3 מ' וגובה 2 מ'.
\[
נפח = 6 \times 3 \times 2 = 36 \text{ מטר מעוקב}
\]

---

## שטח פנים של תיבה

### מהו שטח פנים?
שטח הפנים של תיבה הוא הסכום של שטחי כל ששת הפאות שלה. מכיוון שכל פאה היא מלבן, אנו מחשבים את שטח כל מלבן בנפרד ואז מחברים.

### נוסחה לחישוב שטח פנים
\[
שטח פנים = 2 \times (אורך \times רוחב + אורך \times גובה + רוחב \times גובה)
\]

#### דוגמה מפורטת:
נניח שתיבה בעלת אורך 4 ס\"מ, רוחב 3 ס\"מ וגובה 2 ס\"מ:
1. שטח שתי הפאות הקדמיות (אורך \times רוחב):
   \[
   4 \times 3 = 12
   \]
   יש 2 פאות כאלה, לכן $12 \times 2 = 24$.

2. שטח שתי הפאות הצדדיות (אורך \times גובה):
   \[
   4 \times 2 = 8
   \]
   יש 2 פאות כאלה, לכן $8 \times 2 = 16$.

3. שטח שתי הפאות העליונות והתחתונות (רוחב \times גובה):
   \[
   3 \times 2 = 6
   \]
   יש 2 פאות כאלה, לכן $6 \times 2 = 12$.

4. שטח הפנים הכולל:
   \[
   24 + 16 + 12 = 52 \text{ ס\"מ רבוע}
   \]

---

## יחידות מדידה לנפח

### מהו נפח?
נפח הוא כמות המקום שהגוף תופס בחלל, והוא נמדד ביחידות מעוקבות.

#### דוגמאות ליחידות נפח:
1. **ס\"מ מעוקב ($\text{סמ"ק}$)**: מתאים לחפצים קטנים.
2. **מטר מעוקב ($\text{מ"ק}$)**: מתאים לחדרים או מכולות גדולות.
3. **ליטר**: 1 ליטר = $1000 \text{ סמ"ק}$.

#### המרה בין יחידות:
1. $1 \text{ מ"ק} = 1000 \text{ ליטר}$.
2. $1 \text{ ליטר} = 1000 \text{ סמ"ק}$.

---

## תרגילים לדוגמה

### תרגיל 1:
תיבה בעלת אורך 12 ס\"מ, רוחב 7 ס\"מ וגובה 5 ס\"מ. חשבו את הנפח.

### תרגיל 2:
תיבה בעלת אורך 10 ס\"מ, רוחב 4 ס\"מ וגובה 3 ס\"מ. חשבו את שטח הפנים.

### תרגיל 3:
מיכל מים בנפח 2 מ"ק. כמה ליטרים של מים ניתן למלא במיכל?

---

## פתרונות לתרגילים

### פתרון 1:
\[
נפח = 12 \times 7 \times 5 = 420 \text{ סמ"ק}
\]

### פתרון 2:
\[
שטח פנים = 2 \times (10 \times 4 + 10 \times 3 + 4 \times 3)
\]
\[
= 2 \times (40 + 30 + 12) = 2 \times 82 = 164 \text{ סמ"ר}
\]

### פתרון 3:
\[
2 \text{ מ"ק} = 2 \times 1000 = 2000 \text{ ליטר}
\]

---

## לסיכום

לימוד חישוב נפחים ושימוש ביחידות מדידה מתאים לתחומים רבים בחיי היום-יום, כמו תכנון חללים, לוגיסטיקה ומדעים. הבנת התיבה היא צעד ראשון מצוין להבנת הגיאומטריה התלת-ממדית.
