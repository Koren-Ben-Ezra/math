# שברים ומספרים עשרוניים

## הקדמה לנושא

שברים ומספרים עשרוניים הם חלק בלתי נפרד מהמתמטיקה היומיומית. הם עוזרים לנו לחלק דברים בצורה מדויקת ולייצג ערכים קטנים משלם, כמו מחיר של מוצרים או כמויות מדויקות במתכונים.

### למה זה חשוב?
תארו לעצמכם שאתם חולקים עוגה עם שלושה חברים. איך תוכלו לחלק אותה בצורה הוגנת? כאן נכנסים השברים לתמונה: כל אחד יקבל רבע ($\frac{1}{4}$) מהעוגה. ואם אתם קונים פריט בחנות שעולה 7.5 ש"ח, תצטרכו להשתמש במספר עשרוני כדי להבין את המחיר.

---

## מהו שבר?

### חלקים משלם:
שבר מייצג חלק משלם. הוא כתוב בצורת $\frac{a}{b}$, כאשר:
- $a$ הוא המונה – כמה חלקים יש לנו.
- $b$ הוא המכנה – לכמה חלקים השלם חולק.

#### דוגמה:
אם יש לנו עוגה שחתכנו ל-8 חתיכות, ואכלנו 3, השבר שמייצג את הכמות שאכלנו הוא $\frac{3}{8}$.

### שברים שווים:
שני שברים שווים אם הם מייצגים את אותו חלק מהשלם.
- לדוגמה: $\frac{1}{2} = \frac{2}{4} = \frac{3}{6}$.

---

## מספרים עשרוניים

מספר עשרוני הוא דרך נוספת לייצג שברים, בעזרת נקודה עשרונית.

#### דוגמאות:
- $0.5$ הוא שבר $\frac{1}{2}$.
- $0.25$ הוא שבר $\frac{1}{4}$.

### איך ממירים?
1. שבר למספר עשרוני: חלקו את המונה במכנה.
   - לדוגמה: $\frac{3}{4} = 3 \div 4 = 0.75$.
2. מספר עשרוני לשבר: כתבו את המספר העשרוני כשבר לפי מספר הספרות אחרי הנקודה.
   - לדוגמה: $0.75 = \frac{75}{100}$ ופשטו: $\frac{75}{100} = \frac{3}{4}$.

---

## פעולות בשברים

### חיבור שברים

#### מהו מכנה משותף?
מכנה משותף הוא מספר שמתחלק בדיוק בכל המכנים של השברים. כשיש שברים בעלי מכנים שונים, אנחנו משתמשים במכנה משותף כדי להפוך את השברים לבעלי אותו מכנה.

##### דוגמה למכנה משותף:
ניקח את השברים $\frac{1}{3}$ ו-$\frac{1}{6}$. המכנה המשותף של 3 ו-6 הוא 6, כי 6 הוא המספר הקטן ביותר שמתחלק גם ב-3 וגם ב-6.

#### איך לחבר שברים?
1. מצאו את המכנה המשותף.
2. הרחיבו כל שבר כך שהמכנה יהיה שווה למכנה המשותף.
3. חברו את המונים ושמרו על המכנה המשותף.

##### דוגמה מפורטת:
נחשב $\frac{2}{5} + \frac{3}{10}$:
1. המכנה המשותף הקטן ביותר ל-5 ול-10 הוא 10.
2. הרחבת $\frac{2}{5}$: $\frac{2}{5} = \frac{4}{10}$.
3. חיבור המונים: $\frac{4}{10} + \frac{3}{10} = \frac{7}{10}$.

##### דוגמה נוספת:
$\frac{1}{4} + \frac{1}{2}$:
1. המכנה המשותף הוא 4.
2. הרחבת $\frac{1}{2}$: $\frac{1}{2} = \frac{2}{4}$.
3. חיבור המונים: $\frac{1}{4} + \frac{2}{4} = \frac{3}{4}$.

---

### חיסור שברים

#### איך לחסר שברים?
1. מצאו את המכנה המשותף.
2. הרחיבו כל שבר כך שהמכנה יהיה שווה למכנה המשותף.
3. הפחיתו את המונים ושמרו על המכנה המשותף.

##### דוגמה מפורטת:
נחשב $\frac{3}{4} - \frac{1}{6}$:
1. המכנה המשותף הקטן ביותר ל-4 ול-6 הוא 12.
2. הרחבת $\frac{3}{4}$: $\frac{3}{4} = \frac{9}{12}$.
   הרחבת $\frac{1}{6}$: $\frac{1}{6} = \frac{2}{12}$.
3. חיסור המונים: $\frac{9}{12} - \frac{2}{12} = \frac{7}{12}$.

##### דוגמה נוספת:
$\frac{5}{8} - \frac{1}{4}$:
1. המכנה המשותף הוא 8.
2. הרחבת $\frac{1}{4}$: $\frac{1}{4} = \frac{2}{8}$.
3. חיסור המונים: $\frac{5}{8} - \frac{2}{8} = \frac{3}{8}$.

---

### כפל שברים

#### איך לכפול שברים?
1. כפל את המונים.
2. כפל את המכנים.
3. פשט את התוצאה אם אפשר.

##### דוגמה מפורטת:
נחשב $\frac{3}{4} \times \frac{2}{5}$:
1. כפל המונים: $3 \times 2 = 6$.
2. כפל המכנים: $4 \times 5 = 20$.
3. התוצאה: $\frac{6}{20} = \frac{3}{10}$ (פישוט).

##### דוגמה נוספת:
$\frac{1}{3} \times \frac{4}{7}$:
1. כפל המונים: $1 \times 4 = 4$.
2. כפל המכנים: $3 \times 7 = 21$.
3. התוצאה: $\frac{4}{21}$.

---

### חילוק שברים

#### איך לחלק שברים?
חילוק שברים מתבצע על ידי הכפלת השבר הראשון בהופכי של השבר השני. ההופכי הוא שבר שבו המונה והמכנה מתהפכים.

##### דוגמה מפורטת:
נחשב $\frac{5}{6} \div \frac{2}{9}$:
1. ההופכי של $\frac{2}{9}$ הוא $\frac{9}{2}$.
2. כפל: $\frac{5}{6} \times \frac{9}{2} = \frac{45}{12}$.
3. פישוט: $\frac{45}{12} = \frac{15}{4}$.

##### דוגמה נוספת:
$\frac{7}{8} \div \frac{3}{4}$:
1. ההופכי של $\frac{3}{4}$ הוא $\frac{4}{3}$.
2. כפל: $\frac{7}{8} \times \frac{4}{3} = \frac{28}{24} = \frac{7}{6}$.

---

## תרגילים לדוגמה

### תרגילים לחיבור וחיסור שברים
1. חשבו את $\frac{3}{8} + \frac{1}{4}$.
2. חשבו את $\frac{5}{12} - \frac{1}{6}$.

### תרגילים לכפל שברים
3. חשבו את $\frac{4}{5} \times \frac{3}{7}$.
4. חשבו את $\frac{2}{3} \times \frac{5}{9}$.

### תרגילים לחילוק שברים
5. חשבו את $\frac{8}{9} \div \frac{4}{3}$.
6. חשבו את $\frac{7}{8} \div \frac{5}{6}$.

---

## פתרונות לתרגילים

1. $\frac{3}{8} + \frac{1}{4} = \frac{3}{8} + \frac{2}{8} = \frac{5}{8}$.
2. $\frac{5}{12} - \frac{1}{6} = \frac{5}{12} - \frac{2}{12} = \frac{3}{12} = \frac{1}{4}$.
3. $\frac{4}{5} \times \frac{3}{7} = \frac{12}{35}$.
4. $\frac{2}{3} \times \frac{5}{9} = \frac{10}{27}$.
5. $\frac{8}{9} \div \frac{4}{3} = \frac{8}{9} \times \frac{3}{4} = \frac{24}{36} = \frac{2}{3}$.
6. $\frac{7}{8} \div \frac{5}{6} = \frac{7}{8} \times \frac{6}{5} = \frac{42}{40} = \frac{21}{20}$.

---

## יישומים בחיי היומיום

1. **מטבח**: חישוב כמויות במתכונים.
2. **קניות**: הבנת מחירים והנחות.
3. **ספורט**: מדידה של זמני משחקים או מרחקים.

---

## לסיכום

שברים ומספרים עשרוניים הם כלי רב עוצמה בכל תחום. ככל שתתרגלו יותר, תרגישו בנוח להשתמש בהם ולפתור בעיות בקלות.
