# מבוא לגיאומטריה בסיסית

## הקדמה לנושא

גיאומטריה היא התחום במתמטיקה שעוסק בצורות, גודלים, ומידות. אנו משתמשים בגיאומטריה כדי להבין את העולם שסביבנו: איך למדוד שטח של חדר? איך לחתוך חתיכת עוגה שווה לכולם? או איך לדעת כמה צבע צריך כדי לצבוע קיר?

### למה זה חשוב?
1. **בנייה ועיצוב**: חישוב שטחים, זוויות ואורכים הם חלק מכל תכנון אדריכלי.
2. **משחקים וספורט**: להבין איך לחשב מסלול של כדור או לקבוע גבולות של מגרש משחקים.
3. **חיי היום-יום**: חישוב כמויות של חומרים, שטחים של גינות, ועוד.

---

## מושגים בסיסיים

### נקודה
נקודה היא המקום הקטן ביותר על מישור. היא מסומנת באות, כמו \"A\" או \"B\".
#### דוגמה:
אם נצייר נקודה על דף, זו תהיה הנקודה \"A\".

### קו
קו הוא אוסף של אינסוף נקודות שמחוברות יחד.
#### סוגי קווים:
1. **קו ישר**: ממשיך לשני הכיוונים בלי סוף.
   - דוגמה: חוט מתוח בין שני עמודים.
2. **קטע**: חלק של קו עם נקודת התחלה ונקודת סיום.
   - דוגמה: קו המחבר בין שתי נקודות בדף.
3. **קרן**: קו שמתחיל בנקודה אחת וממשיך לכיוון אחד בלי סוף.
   - דוגמה: אור שיוצא מפנס.

### זווית
זווית נוצרת כאשר שני קווים נפגשים בנקודה אחת. 
- הנקודה שבה הם נפגשים נקראת \"קודקוד\".
- הקווים שמרכיבים את הזווית נקראים \"שוקיים\".
#### סוגי זוויות:
1. **זווית חדה**: קטנה מ-90 מעלות.
   - דוגמה: זווית של קצה משולש חד.
2. **זווית ישרה**: בדיוק 90 מעלות, כמו פינה של ריבוע.
   - דוגמה: פינה של שולחן מרובע.
3. **זווית קהה**: גדולה מ-90 מעלות וקטנה מ-180 מעלות.
   - דוגמה: פתיחה של דלת שנפתחה לא עד הסוף.
4. **זווית שטוחה**: בדיוק 180 מעלות.
   - דוגמה: קו ישר שנמתח משני הכיוונים.

---

## צורות בסיסיות

### משולש
משולש הוא צורה עם 3 צלעות ו-3 זוויות. 
#### נוסחאות:
- שטח: $\frac{1}{2} \times הבסיס \times הגובה$
- היקף: סכום אורכי שלוש הצלעות.

#### סוגי משולשים:
1. **משולש שווה-צלעות**: כל הצלעות והזוויות שוות.
   - דוגמה: משולש של מתקן תלייה בצורת פירמידה קטנה.
2. **משולש שווה-שוקיים**: שתי צלעות שוות.
   - דוגמה: סימן אזהרה תלת-צדדי.
3. **משולש ישר-זווית**: יש בו זווית אחת של 90 מעלות.
   - דוגמה: פינה של דף מקופל לצורת משולש.

#### דוגמה לחישוב:
משולש עם בסיס באורך 6 ס\"מ וגובה 4 ס\"מ:
- שטח = $\frac{1}{2} \times 6 \times 4 = 12$ ס\"מ רבוע.

---

### ריבוע
ריבוע הוא צורה עם 4 צלעות שוות ו-4 זוויות ישרות.
#### נוסחאות:
- שטח: $צלע \times צלע$
- היקף: $4 \times אורך הצלע$

#### דוגמה לחישוב:
אם אורך הצלע הוא 5 ס\"מ:
- שטח = $5 \times 5 = 25$ ס\"מ רבוע.
- היקף = $4 \times 5 = 20$ ס\"מ.

---

### מלבן
מלבן הוא צורה עם 4 צלעות ושני זוגות של צלעות נגדיות שוות. כל הזוויות בו ישרות.
#### נוסחאות:
- שטח: $אורך \times רוחב$
- היקף: $2 \times (אורך + רוחב)$

#### דוגמה לחישוב:
אם אורך המלבן הוא 8 ס\"מ ורוחבו 3 ס\"מ:
- שטח = $8 \times 3 = 24$ ס\"מ רבוע.
- היקף = $2 \times (8 + 3) = 22$ ס\"מ.

---

### עיגול
עיגול הוא צורה שבה כל הנקודות במישור נמצאות במרחק שווה מנקודה מרכזית.

#### מהו פאי ($\pi$)?
פאי ($\pi$) הוא מספר מיוחד במתמטיקה שמייצג את היחס בין היקף המעגל לקוטרו. הערך של פאי בקירוב הוא $3.14$. הוא חשוב כי הוא מאפשר לנו לחשב היקפים ושטחים של מעגלים בדיוק.

#### נוסחאות:
- שטח: $\pi \times רדיוס^2$
- היקף: $2 \pi \times רדיוס$

#### דוגמה לחישוב:
אם רדיוס העיגול הוא 7 ס\"מ:
- היקף = $2 \pi \times 7 = 2 \times 3.14 \times 7 = 43.96$ ס\"מ.
- שטח = $\pi \times 7^2 = 3.14 \times 49 = 153.86$ ס\"מ רבוע.

---

## תרגילים לדוגמה

### תרגיל 1:
ציירו משולש עם בסיס באורך 10 ס\"מ וגובה 5 ס\"מ. חשבו את השטח.

**פתרון**: 
- שטח = $\frac{1}{2} \times 10 \times 5 = 25$ ס\"מ רבוע.

### תרגיל 2:
אם לריבוע יש צלע באורך 4 ס\"מ, חשבו את השטח וההיקף.

**פתרון**: 
- שטח = $4 \times 4 = 16$ ס\"מ רבוע.
- היקף = $4 \times 4 = 16$ ס\"מ.

### תרגיל 3:
מלבן אורכו 9 ס\"מ ורוחבו 2 ס\"מ. חשבו את השטח וההיקף.

**פתרון**: 
- שטח = $9 \times 2 = 18$ ס\"מ רבוע.
- היקף = $2 \times (9 + 2) = 22$ ס\"מ.

### תרגיל 4:
עיגול עם רדיוס של 5 ס\"מ. חשבו את ההיקף והשטח (עגלו את התוצאה ל-2 ספרות אחרי הנקודה).

**פתרון**: 
- היקף = $2 \pi \times 5 = 31.4$ ס\"מ.
- שטח = $\pi \times 5^2 = 78.5$ ס\"מ רבוע.

---

## לסיכום

גיאומטריה היא תחום מרתק שמלמד אותנו איך לחשב, למדוד ולהבין צורות במציאות. באמצעות התרגול, תוכלו להרגיש בנוח לעבוד עם זוויות, שטחים והיקפים בכל מצב!
