# נושא 8: פונקציות בסיסיות

## הקדמה לנושא

פונקציות הן דרך לתאר קשרים בין שני דברים. אפשר לחשוב על זה כעל "חוק מתמטי" שמחבר ערך אחד (קלט) לערך אחר (פלט). 
למשל, אם אנחנו קונים תפוחים, המחיר נקבע לפי מספר התפוחים. פונקציות עוזרות לנו להבין איך דברים משתנים כאשר אנחנו משנים ערכים שונים. 

פונקציות מופיעות בכל מקום סביבנו – בחישובי מרחק בזמן נסיעה, במדידת מהירות או בטמפרטורות, ואפילו בעיצוב של משחקי מחשב. 

### למה זה חשוב?
1. **מתמטיקה**: פונקציות הן הבסיס לנושאים מורכבים יותר כמו אלגברה, גיאומטריה אנליטית, ואף משוואות מתקדמות.
2. **חיי היום-יום**: הן עוזרות לנו לחשב כמויות, עלויות ומדידות שונות בצורה קלה וברורה.
3. **מדע וטכנולוגיה**: תופעות פיזיקליות כמו מהירות, חום, זמן, כוח וכדומה מתוארות פעמים רבות באמצעות פונקציות.

---

## מהי פונקציה?

### הגדרה
פונקציה היא "כלל" או "חוק" שמקשר בין ערך מסוים שנקרא "קלט" לערך אחר שנקרא "פלט". 
בפונקציה אמיתית, לכל ערך של הקלט יש בדיוק ערך אחד של הפלט. 

#### איך אפשר לחשוב על פונקציה?
דמיינו שיש לכם מכונה מיוחדת:
- אתם מכניסים לתוכה ערך כלשהו (זהו הקלט).
- המכונה מבצעת פעולה (כמו חיבור, חיסור או כל פעולה אחרת).
- המכונה מחזירה לכם ערך חדש (זהו הפלט).

כל פעם שמכניסים את אותו הקלט, המכונה תחזיר את אותו הפלט.

#### דוגמה פשוטה:
נניח שיש לנו מכונה שמכפילה את המספר שנכניס ב-2. 
- אם נכניס 3, נקבל 6. 
- אם נכניס 5, נקבל 10.

אפשר לכתוב את זה בצורה מתמטית:
$$
 y = 2x 
$$
כאן:
- $x$ הוא הקלט.
- $y$ הוא הפלט.

#### דוגמה מהחיים:
אם קונים תפוחים במחיר של 3 שקלים לכל תפוח:
- הקלט: מספר התפוחים.
- הפלט: המחיר הכולל.

הפונקציה תהיה:
$$
 מחיר = 3 \times \text{מספר התפוחים} 
$$

---

## מושגים חשובים בפונקציות

### תחום ההגדרה
"תחום ההגדרה" הוא קבוצת הערכים שיכולים להיכנס לפונקציה כקלט. 
למשל, אם מדברים על מספר התפוחים שאפשר לקנות, תחום ההגדרה יכול להיות כל המספרים הטבעיים (1, 2, 3, …).

### טווח
"טווח" הוא קבוצת הערכים שהפונקציה יכולה להחזיר כפלט. 
בדוגמת התפוחים, אם הקלט הוא מספר התפוחים, הטווח יהיה ערכי המחיר שהפונקציה יכולה להוציא (3, 6, 9, …).

---

## ייצוג של פונקציות

ניתן לתאר פונקציות בכמה דרכים:

### 1. טבלה
בטבלה אנחנו רושמים שורה של ערכי קלט ($x$) ושורה מקבילה של ערכי פלט ($y$). 

#### דוגמה:
נניח שהמחיר של כל תפוח הוא 3 שקלים:

| מספר התפוחים ($x$) | המחיר ($y$) |
|------------------|-----------|
| 1                | 3         |
| 2                | 6         |
| 3                | 9         |
| 4                | 12        |

### 2. גרף
גרף מציג את הקשר בין הקלט לפלט על מערכת צירים: 
- ציר ה-$x$ מייצג את הקלט.
- ציר ה-$y$ מייצג את הפלט.

לדוגמה, פונקציה $y = 3x$ תופיע כקו ישר שעובר דרך (1,3), (2,6), (3,9) וכן הלאה.

### 3. ביטוי אלגברי
ביטוי אלגברי הוא נוסחה קצרה המראה בדיוק את הקשר. 
למשל, $y = 3x$ אומר שלכל $x$, ערך $y$ יהיה 3 כפול $x$. 
אם $x = 2$, אז $y = 6$; אם $x = 5$, אז $y = 15$.

---

## סוגים של פונקציות בסיסיות

### 1. פונקציות קוויות
פונקציות שבהן הקשר בין הקלט לפלט יוצר קו ישר בגרף. אם נשים ערכים בטבלה או נשרטט בגרף, נראה שהשינוי הוא אחיד.

#### דוגמה:
$$
 y = 2x + 1 
$$

- כאשר $x = 0$, $y = 1$.
- כאשר $x = 1$, $y = 3$.
- כאשר $x = 2$, $y = 5$.

בגרף, כל הנקודות הללו יושבות על קו ישר אחד.

### 2. פונקציות קבועות
פונקציות שבהן הפלט לא משתנה, לא משנה מהו הקלט.

#### דוגמה:
$$
 y = 5 
$$

- אם $x = 0$, $y = 5$.
- אם $x = 1$, $y = 5$.
- אם $x = 10$, $y = 5$.

תמיד נקבל את אותו פלט, ולכן הגרף נראה כקו אופקי ישר.

### 3. פונקציות מדורגות (שלבים)
פונקציות שבהן הפלט משתנה בקפיצות שלמות, בדרך כלל במדרגות.

#### דוגמה:
בחנות יש מבצע: כל 3 תפוחים עולים 10 שקלים. 
- אם קונים 1, 2 או 3 תפוחים → המחיר הוא 10 שקלים.
- אם קונים 4, 5 או 6 תפוחים → המחיר הוא 20 שקלים.
- אם קונים 7, 8 או 9 תפוחים → המחיר הוא 30 שקלים.

אם נצייר את זה בגרף, יתקבלו "מדרגות" שעולות בקפיצות.

---

## תרגילים לדוגמה

### תרגיל 1:
מצאו את ערכי הפלט של הפונקציה $y = 4x$ עבור הערכים $x = 1, 2, 3, 4$.

### תרגיל 2:
ציירו טבלה עבור הפונקציה $y = x + 2$ עבור $x = 0, 1, 2, 3$.

### תרגיל 3:
ציירו גרף עבור הפונקציה $y = 2x + 1$. כתבו את הנקודות העיקריות שעל הגרף.

### תרגיל 4:
האם הפונקציה $y = 7$ היא פונקציה קבועה? הסבירו מדוע.

---

## פתרונות לתרגילים

### פתרון 1:
| $x$ | $y = 4x$ |
|---------|-------------|
| 1       | 4           |
| 2       | 8           |
| 3       | 12          |
| 4       | 16          |

### פתרון 2:
| $x$ | $y = x + 2$ |
|---------|----------------|
| 0       | 2              |
| 1       | 3              |
| 2       | 4              |
| 3       | 5              |

### פתרון 3:
למשל, אפשר לחשב נקודות:
- $x = 0 \rightarrow y = 2 \times 0 + 1 = 1  \rightarrow  נקודה (0, 1)$
- $x = 1 \rightarrow y = 2 \times 1 + 1 = 3  \rightarrow  נקודה (1, 3)$
- $x = 2 \rightarrow y = 2 \times 2 + 1 = 5  \rightarrow  נקודה (2, 5)$
- $x = 3 \rightarrow y = 2 \times 3 + 1 = 7  \rightarrow  נקודה (3, 7)$

נצייר את הנקודות ונקבל קו ישר.

### פתרון 4:
כן, $y = 7$ היא פונקציה קבועה, כי עבור כל ערך של $x$, נקבל פלט של 7.

---

## לסיכום

פונקציות הן רעיון מרכזי במתמטיקה ועוזרות לנו לתאר קשרים רבים בעולם: החל ממחיר של מוצרים בחנות ועד למדידת גדילה בטבע. 
אנו יכולים להציג פונקציות בדרכים שונות: טבלה, גרף או נוסחה. 
למדנו על כמה סוגים בסיסיים של פונקציות: קוויות, קבועות ומדורגות. 
באמצעות הבנה של תחום ההגדרה (הקלטים האפשריים) והטווח (הפלטים האפשריים), אנחנו יכולים להבין טוב יותר את "החוק" שמתאר את הקשר.

ככל שנעמיק בעולם הפונקציות, נוכל לפתור בעיות מורכבות יותר ולהשתמש בכלים האלה בתחומי מדע, טכנולוגיה וכמובן גם בחיי היומיום שלנו.
