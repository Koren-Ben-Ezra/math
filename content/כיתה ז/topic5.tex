# משוואות בנעלם אחד

## הקדמה לנושא

משוואה היא משפט מתמטי שאומר ששני ביטויים הם שווים. כאשר במשוואה יש משתנה (סימן שמייצג מספר לא ידוע), המטרה היא למצוא את הערך של המשתנה שהופך את המשפט לנכון.

### למה זה חשוב?
פתרון משוואות בנעלם אחד הוא הבסיס לכל המתמטיקה המתקדמת. זה מאפשר לנו:
1. להבין קשרים בין מספרים.
2. לפתור בעיות יומיומיות, כמו חישוב מחירים או כמויות.
3. להכין אותנו לנושאים מתקדמים יותר כמו אלגברה וגיאומטריה אנליטית.

---

## מהי משוואה?

### הגדרה
משוואה היא ביטוי מתמטי שמכיל סימן שוויון ($=$). לדוגמה:
- $3x + 2 = 11$

### מהו נעלם?
נעלם הוא משתנה במשוואה, המיוצג לרוב באותיות כמו $x$, $y$ או $z$. המטרה שלנו היא למצוא את הערך של הנעלם.

#### דוגמה:
במשוואה $x + 3 = 7$, הנעלם הוא $x$. הפתרון הוא $x = 4$, כי $4 + 3 = 7$.

---

## שלבים לפתרון משוואה בנעלם אחד

### שלב 1: פשטו את המשוואה
אם ישנם איברים משני צדי המשוואה, נסו להפוך את המשוואה לפשוטה יותר על ידי חיבור, חיסור, כפל או חילוק.

#### דוגמה:
במשוואה $3x + 2 = 11$:
- נחסיר $2$ משני צדי המשוואה: $3x = 9$.

### שלב 2: הפרידו את הנעלם
נסו להשאיר את הנעלם לבד בצד אחד של המשוואה.

#### דוגמה:
המשוואה הפכה ל-$3x = 9$.
- נחלק את שני הצדדים ב-$3$: $x = 3$.

### שלב 3: בדקו את הפתרון
הציבו את הפתרון במשוואה המקורית ובדקו אם התוצאה נכונה.

#### דוגמה:
- נבדוק: $3 \times 3 + 2 = 11$. נכון! הפתרון הוא $x = 3$.

---

## סוגים שונים של משוואות

### משוואות עם חיבור או חיסור

#### דוגמה מפורטת:
משוואה: $x + 5 = 12$.
1. נחסיר $5$ משני הצדדים: $x = 12 - 5$.
2. $x = 7$.
3. בדיקה: $7 + 5 = 12$. נכון!

#### דוגמה נוספת:
משוואה: $x - 4 = 9$.
1. נוסיף $4$ לשני הצדדים: $x = 9 + 4$.
2. $x = 13$.

### משוואות עם כפל או חילוק

#### דוגמה מפורטת:
משוואה: $4x = 20$.
1. נחלק את שני הצדדים ב-$4$: $x = 20 \div 4$.
2. $x = 5$.
3. בדיקה: $4 \times 5 = 20$. נכון!

#### דוגמה נוספת:
משוואה: $\frac{x}{3} = 6$.
1. נכפיל את שני הצדדים ב-$3$: $x = 6 \times 3$.
2. $x = 18$.

### משוואות עם שני שלבים

#### דוגמה מפורטת:
משוואה: $2x + 3 = 11$.
1. נחסיר $3$ משני הצדדים: $2x = 8$.
2. נחלק ב-$2$: $x = 8 \div 2$.
3. $x = 4$.

#### דוגמה נוספת:
משוואה: $\frac{x}{2} - 5 = 3$.
1. נוסיף $5$ לשני הצדדים: $\frac{x}{2} = 8$.
2. נכפיל ב-$2$: $x = 16$.

---

## תרגילים לדוגמה

### תרגילים בסיסיים:
1. $x + 7 = 15$
2. $x - 3 = 10$
3. $5x = 25$
4. $\frac{x}{4} = 9$

### תרגילים מורכבים יותר:
5. $3x + 2 = 17$
6. $\frac{x}{5} - 4 = 2$
7. $7x - 5 = 30$
8. $2x + 3x = 25$

---

## פתרונות לתרגילים

### פתרונות בסיסיים:
1. $x + 7 = 15 \implies x = 15 - 7 \implies x = 8$
2. $x - 3 = 10 \implies x = 10 + 3 \implies x = 13$
3. $5x = 25 \implies x = 25 \div 5 \implies x = 5$
4. $\frac{x}{4} = 9 \implies x = 9 \times 4 \implies x = 36$

### פתרונות מורכבים יותר:
5. $3x + 2 = 17 \implies 3x = 17 - 2 \implies 3x = 15 \implies x = 15 \div 3 \implies x = 5$
6. $\frac{x}{5} - 4 = 2 \implies \frac{x}{5} = 2 + 4 \implies \frac{x}{5} = 6 \implies x = 6 \times 5 \implies x = 30$
7. $7x - 5 = 30 \implies 7x = 30 + 5 \implies 7x = 35 \implies x = 35 \div 7 \implies x = 5$
8. $2x + 3x = 25 \implies 5x = 25 \implies x = 25 \div 5 \implies x = 5$

---

## לסיכום

פתרון משוואות בנעלם אחד הוא כלי חשוב במתמטיקה ובחיי היום-יום. על ידי תרגול והבנה של השלבים, תוכלו לפתור כל משוואה בקלות ובביטחון!
