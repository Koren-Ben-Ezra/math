# כיתה ח' – משוואות קוויות בנעלם אחד

### הצצה לעולם המשוואות המתקדמות

דמיינו שאתם מתקדמים בשלבי משחק שבו כל חידה שאתם פותרים הופכת למורכבת יותר. עכשיו אנחנו מתקדמים למשוואות קוויות בנעלם אחד, שדורשות מעט יותר מחשבה אך גם יותר סיפוק כשפותרים אותן.

---

### למה זה חשוב? (מוטיבציה חמודה)

נניח שאתם מנסים לחשב כמה עולה להזמין פיצה ל-5 חברים, כשלכל אחד מהם תוספות שונות. כדי לחשב בצורה מסודרת, תצטרכו להשתמש במשוואה שבה יש יותר שלבים. בעזרת משוואות קוויות, תוכלו להגיע לתשובה המדויקת במהירות!

---

## נושא 3: משוואות קוויות

### מה זה משוואה קווית?

משוואה קווית היא משוואה שבה הנעלם מופיע בלי חזקה (לדוגמה, $x$, ולא $x^2$). המשוואה בדרך כלל נראית כך: $ax + b = c$, כש-$a$, $b$, ו-$c$ הם מספרים.

---

### דוגמה פתורה:

**השאלה:**
פתרו את המשוואה $2x + 5 = 15$.

**פתרון:**

1. נבודד את $x$ על ידי חיסור 5 משני הצדדים:
   $2x + 5 - 5 = 15 - 5$
   $$
   2x = 10
   $$
2. נחלק את שני הצדדים ב-2:
   $2x \div 2 = 10 \div 2$
   $$
   x = 5
   $$

**תשובה:** $x = 5$

---

### תרגילים פתורים:

1. **השאלה:**
פתרו את המשוואה $3x - 4 = 14$.

**פתרון:**

- נוסיף 4 לשני הצדדים:
  $3x - 4 + 4 = 14 + 4$
  $$
  3x = 18
  $$
- נחלק ב-3:
  $3x \div 3 = 18 \div 3$
  $$
  x = 6
  $$

**תשובה:** $x = 6$

2. **השאלה:**
פתרו את המשוואה $5x + 7 = 2x + 16$.

**פתרון:**

- נחסר $2x$ משני הצדדים:
  $5x - 2x + 7 = 2x - 2x + 16$
  $$
  3x + 7 = 16
  $$
- נחסר 7 משני הצדדים:
  $3x + 7 - 7 = 16 - 7$
  $$
  3x = 9
  $$
- נחלק ב-3:
  $3x \div 3 = 9 \div 3$
  $$
  x = 3
  $$

**תשובה:** $x = 3$

---

### תרגילים:

1. פתרו את המשוואה $4x - 6 = 10$.
   - נוסיף 6: $4x = 16$
   - נחלק ב-4: $x = 4$
   **תשובה:** $x = 4$

2. פתרו את המשוואה $7x + 2 = 5x + 12$.
   - נחסר $5x$: $2x + 2 = 12$
   - נחסר 2: $2x = 10$
   - נחלק ב-2: $x = 5$
   **תשובה:** $x = 5$

3. פתרו את המשוואה $6x - 8 = 2x + 4$.
   - נחסר $2x$: $4x - 8 = 4$
   - נוסיף 8: $4x = 12$
   - נחלק ב-4: $x = 3$
   **תשובה:** $x = 3$

---

## לסיכום:

משוואות קוויות עוזרות לנו לפתור בעיות מורכבות יותר, כמו חישובי עלויות או חלוקת משאבים. כשתתרגלו את השלבים, תגלו שזה הופך להיות קל ומהנה.

### רוצים להמשיך לשלב הבא? בואו נצלול לעומק נוסף!

